%% ****** Start of file apstemplate.tex ****** %
%%
%%
%%   This file is part of the APS files in the REVTeX 4.2 distribution.
%%   Version 4.2a of REVTeX, January, 2015
%%
%%
%%   Copyright (c) 2015 The American Physical Society.
%%
%%   See the REVTeX 4 README file for restrictions and more information.
%%
%
% This is a template for producing manuscripts for use with REVTEX 4.2
% Copy this file to another name and then work on that file.
% That way, you always have this original template file to use.
%
% Group addresses by affiliation; use superscriptaddress for long
% author lists, or if there are many overlapping affiliations.
% For Phys. Rev. appearance, change preprint to twocolumn.
% Choose pra, prb, prc, prd, pre, prl, prstab, prstper, or rmp for journal
%  Add 'draft' option to mark overfull boxes with black boxes
%  Add 'showkeys' option to make keywords appear


\documentclass[reprint,aps,prb,citeautoscript,altaffilletter]{revtex4-2}

%----------------------------------------------------------------------------------------
%
%----------------------------------------------------------------------------------------


%----------------------------------------------------------------------------------------
%	Paquetes
%----------------------------------------------------------------------------------------
\usepackage[activeacute,spanish,mexico,es-tabla]{babel}
\usepackage[utf8]{inputenc}
\usepackage[T1]{fontenc}
\usepackage{latexsym}
\usepackage{lipsum}
%----------------------------------------------------------------------------------------
%
%----------------------------------------------------------------------------------------




\begin{document}
	
	
	% -----> TITLE PAGE
	\title{ A development of a Near-Field Optical Microcope }
	\author{\underline{G. A. Mart\'inez-Zepeda}}
	\email{gabmtzz27@gmail.com}
	\author{O. Ruiz-Cigarrillo}
	\email{oscarruiz@cactus.iico.uaslp.mx}
	\author{{K. P. Leija-Alf\'erez}}
	\email{a236792@alumnos.uaslp.mx}
	\author{L. F. Lastras-Mart\'inez}
	\email{lflm@cactus.iico.uaslp.mx}
	\affiliation{Instituto de Investigacion en Comunicaci\'on \'Optica, Universidad Autonoma de San Luis Potos\'i}
	% --->   DATE
	
	\date{\today}
	
	% -----> ABSTRACT	
	\begin{abstract}
		There exist a lot of Scanning Probe Microscopy Systems, such as Atomic Force Microscopy (AFM) that use a metallic tip and are used to get a image with nanometric resolution, but they are expensive  and in some cases need special conditions for operation, in other hand, the Near-Field Scanning Optical Microscopy (NSOM) systems use a dielectric tip
		and the interaction between the tip and the sample is due optical processes, it makes posible their operation under ambient conditions. In this work we show the first results of the development  of a Near-Field Microscope where its artisanal construction gives it some advantages from comercial ones.
		
	\end{abstract}
	
	\maketitle
	

	
%	\bibliographystyle{apsrev4-2}
%	\bibliography{./REF.bib}
%	\nocite{*}
	
\end{document}

