%% ****** Start of file apstemplate.tex ****** %
%%
%%
%%   This file is part of the APS files in the REVTeX 4.2 distribution.
%%   Version 4.2a of REVTeX, January, 2015
%%
%%
%%   Copyright (c) 2015 The American Physical Society.
%%
%%   See the REVTeX 4 README file for restrictions and more information.
%%
%
% This is a template for producing manuscripts for use with REVTEX 4.2
% Copy this file to another name and then work on that file.
% That way, you always have this original template file to use.
%
% Group addresses by affiliation; use superscriptaddress for long
% author lists, or if there are many overlapping affiliations.
% For Phys. Rev. appearance, change preprint to twocolumn.
% Choose pra, prb, prc, prd, pre, prl, prstab, prstper, or rmp for journal
%  Add 'draft' option to mark overfull boxes with black boxes
%  Add 'showkeys' option to make keywords appear


\documentclass[reprint,aps,prb,citeautoscript,altaffilletter]{revtex4-2}

%----------------------------------------------------------------------------------------
%
%----------------------------------------------------------------------------------------


%----------------------------------------------------------------------------------------
%	Paquetes
%----------------------------------------------------------------------------------------
\usepackage[activeacute,spanish,mexico,es-tabla]{babel}
\usepackage[utf8]{inputenc}
\usepackage[T1]{fontenc}
\usepackage{latexsym}
\usepackage{lipsum}
%----------------------------------------------------------------------------------------
%
%----------------------------------------------------------------------------------------




\begin{document}
	
	
	% -----> TITLE PAGE
	\title{Spin resolve Reflectance Anisotropy Spectroscopy }
	\author{\underline{M. C. Rangel-Monreal}}
	\author{\underline{O. Ruiz-Cigarrillo}}
	\email{oscarruiz@cactus.iico.uaslp.mx}
	\author{G. Flores-Rangel}
	\author{G. A. Mart\'inez-Zepeda }
	\author{C. A. Bravo-Vel\'azquez}
	\author{R. E. Balderas-Navarro}
	\author{L. F. Lastras-Mart\'inez}
	\email{lflm@cactus.iico.uaslp.mx}
	\affiliation{Instituto de Investigacion en Comunicaci\'on \'Optica, Universidad Autonoma de San Luis Potos\'i}
	% --->   DATE
	
	\date{\today}
	
	% -----> ABSTRACT	
	\begin{abstract}
		This work present a novel source to expand the power of Reflectance Anisotropy Spectroscopy (RAS). The RAS is a technique usefully used to study optical properties in semiconductors, this work starti
		ng from the study of Coupled Quantum Wells structures (CQWs), these structures has the characteristic of  increases the in-plane anisotropy due to the relative width between the wells (this means, one of the well is wider than the other). This is because of the symmetry reduction, therefore is expected which these structures exhibit important physical phenomena as spin properties. This work raise the RAS setup modify to get spin response.   
	\end{abstract}
	
	\maketitle
	

	
%\bibliographystyle{apsrev4-2}
%\bibliography{./REF.bib}
%\nocite{*}
	
\end{document}
